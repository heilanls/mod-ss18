\documentclass{article}

\input{usepackage.tex}

\begin{document}
	%%% Vorgegebenes Deckblatt %%%
	\includepdf{deckblatt.pdf}
	
	%%% format and header %%%
	% Counter für das Blatt und die Aufgabennummer.
% Ersetze die Nummer des Übungsblattes und die Nummer der Aufgabe
% den Anforderungen entsprechend.
% Beachte:
% \setcounter{countername}{number}: Legt den Wert des Counters fest
% \stepcounter{countername}: Erhöht den Wert des Counters um 1.
\newcounter{sheetnr}
\setcounter{sheetnr}{2} % Nummer des Übungsblattes
\newcounter{exnum}
\setcounter{exnum}{1} % Nummer der Aufgabe

% Befehl für die Aufgabentitel
\newcommand{\exercise}[1]{\section*{Aufgabe \theexnum\stepcounter{exnum} #1}} % Befehl für Aufgabentitel

% Formatierung der Kopfzeile
% \ohead: Setzt rechten Teil der Kopfzeile mit
% Namen und Matrikelnummern aller Bearbeiter
\ohead{Yannis Blosch (3256958)\\
Lukas Heiland (3269754)}
% \chead{} kann mittleren Kopfzeilen Teil sezten
% \ihead: Setzt linken Teil der Kopfzeile mit
% Modulnamen, Semester und Übungsblattnummer
\ihead{Modellierung\\
Sommersemester 2018\\
Blatt \thesheetnr}
	
	%%%%%%%%%%%%%%%%%%%%%%%%%%%%%%
	%%%%%%% actual content %%%%%%%
	%%%%%%%%%%%%%%%%%%%%%%%%%%%%%%
	
	\section*{Aufgabe 4.1}
		% relationale Algebra: Datum aller Darlehen über >5.000€
		\paragraph*{a.}
			TODO
			
		% rel. Algebra: IDs aller Konten die ein Darl. über >20.000€ haben und KOntaktperson in Berlin
		\paragraph*{b.}
			TODO
			
		% SQL-query: Name,Alter,Anzahl Kredite,PLZ aller Kunden absteigend nach Alter sortiert
		\paragraph*{c.}
			TODO
			
		% Geben Sie eine Anfrage in SQL-Notation an, die für jeden Mitarbeiter den Namen und die Anzahl der Bankkonten, bei denen er als Kontaktperson fungiert und deren Guthaben niedriger als 5000 ist, ausgibt. Es sollen nur Mitarbeiter ausgegeben werden, die insgesamt bei weniger als 50 Bankkonten Kontaktpersonen sind.
		\paragraph*{d.}
			TODO
		
		% SQL-query: Name,Alter aller Kunden, die mind. 1 Darl. haben, dessen Höhe > Durchschnittskredithöhe ist (ohne Duplikate -> DISTINCT)
		\paragraph*{e.}
			TODO
		
		% SQL: immer wenn in einer Adresse "Stuttgart" vorkommt, soll der Ort in "Stuttgart" geändert werden. z.B.: Stuttgart-Vaihingen -> Stuttgart, PenisStuttgart -> Stuttgart
		\paragraph*{f.}
			TODO
	
		\pagebreak
	
	\section*{Aufgabe 4.2}
\end{document}