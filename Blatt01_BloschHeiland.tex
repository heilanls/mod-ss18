\documentclass{article}

\input{usepackage.tex}

\begin{document}
	\includepdf[pages=-]{deckblatt.pdf}
	
	% Counter für das Blatt und die Aufgabennummer.
% Ersetze die Nummer des Übungsblattes und die Nummer der Aufgabe
% den Anforderungen entsprechend.
% Beachte:
% \setcounter{countername}{number}: Legt den Wert des Counters fest
% \stepcounter{countername}: Erhöht den Wert des Counters um 1.
\newcounter{sheetnr}
\setcounter{sheetnr}{2} % Nummer des Übungsblattes
\newcounter{exnum}
\setcounter{exnum}{1} % Nummer der Aufgabe

% Befehl für die Aufgabentitel
\newcommand{\exercise}[1]{\section*{Aufgabe \theexnum\stepcounter{exnum} #1}} % Befehl für Aufgabentitel

% Formatierung der Kopfzeile
% \ohead: Setzt rechten Teil der Kopfzeile mit
% Namen und Matrikelnummern aller Bearbeiter
\ohead{Yannis Blosch (3256958)\\
Lukas Heiland (3269754)}
% \chead{} kann mittleren Kopfzeilen Teil sezten
% \ihead: Setzt linken Teil der Kopfzeile mit
% Modulnamen, Semester und Übungsblattnummer
\ihead{Modellierung\\
Sommersemester 2018\\
Blatt \thesheetnr}
	
	\section*{Aufgabe 1.1}
		\paragraph*{a.} Falsch, denn die Kardinalität eines Kapitels ist 1 (eine "1" auf der Kante zwischen $Buch$ und $enthaelt$) $\Rightarrow$ Ein Kapitel kann nur in einem Buch enthalten sein.
		
		\paragraph*{b.} Richtig. Eine Kardinalität von n bedeutet in Min/Max-Notation (0,*).
		
		\paragraph*{c.} Falsch, Kapitel können auf derselben Seite beginnen, das Attribut $Startseite$ ist nicht unique.
		
		\paragraph*{d.} Ja, dies ist sinnvoll, denn Kapitel können ohne Buch nicht existieren und lässt sich auch nur mit Hilfe von ihm klar identifizieren.
		
		\paragraph*{e.}Falsch, er kann in (0,*) Fächern Prüfungen abnehmen. Das schließen wir aus dem m.
		
		\paragraph*{f.}
		
		\paragraph*{g.}Richtig, ein Prüfer kann einen Studenten, denn in n (0,*) ist die 1 enthalten, in m, haben wir schon festgestellt ist beliebig, Fächern testen. 
		
		\paragraph*{h.}Falsch, das Attribut Datum ist nicht unique.
		
		\paragraph*{i.}Es sind Kurse mit nur drei Teilnehmern aber nicht beliebig vielen, minimal ein Teilnehmer, maximal 5
		
		\paragraph*{j.}Falsch, sie können 10 oder mehr Kurse besuchen.
		
		\paragraph*{k.}Falsch, sie müssen mindestens 10 besuchen.
		
		\paragraph*{l.}Falsch, sie muss immer angegeben werden.
		%%%%%%%%%%%%%%%%%%%%% TODO: Aufgabenteile d) - l) %%%%%%%%%%%%%%%%%%%%%%%%%
		%%%%%%%%%%%%%%%%%%%%%%%%%%%%%%%%%%%%%%%%%%%%%%%%%%%%%%%%%%%%%%%%%%%%%%%%%%%
		
	\section*{Aufgabe 1.2}
		%%% Aufgabenteil a (LHE)
		\subsection*{a.}
		.
			\begin{figure}[h]
				\includegraphics[width=0.7\textwidth]{aufgabe_1_2_a.png}
			\end{figure}
		
		%%% Aufgabenteil b (YBL)
		\subsection*{b.}
		.
			\begin{figure}[h]
				\includegraphics[width=0.7\textwidth]{aufgabe_1_2_b.png}
			\end{figure}
		
		%%% Aufgabenteil c (LHE)
		\subsection*{c.}
		.
			\begin{figure}[h]
				\includegraphics[width=0.7\textwidth]{aufgabe_1_2_c.png}
			\end{figure}
			
		
		%%% Aufgabenteil d (YBL)
		\subsection*{d.}
		.
	%%%		\begin{figure}[h]
	%%%				\includegraphics[width=0.7\textwidth]{xxx.png}
	%%%			\end{figure}
		
		%%% Aufgabenteil e (LHE)
		\subsection*{e.}
		.
			\begin{figure}[h]
				\includegraphics[width=0.7\textwidth]{aufgabe_1_2_e.png}
			\end{figure}
		
		%%% Aufgabenteil f (YBL)
		\subsection*{f.}
		.
		%%%		\begin{figure}[h]
		%%%			\includegraphics[width=0.7\textwidth]{xxx.png}
		%%%		\end{figure}
		
		
		%%% layout reasons
		\pagebreak
		
		%%% Aufgabenteil g (LHE)
		\subsection*{g.}
		.
			\begin{figure}[h]
				\includegraphics[width=0.7\textwidth]{aufgabe_1_2_g.png}
			\end{figure}
		
		%%% Aufgabenteil h (YBL)
		\subsection*{h.}
		.
		%%%		\begin{figure}[h]
		%%%			\includegraphics[width=0.7\textwidth]{xxx.png}
		%%%		\end{figure}
		
\end{document}